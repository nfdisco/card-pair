\documentclass[a4paper,12pt]{article}

\usepackage{amsmath}
\usepackage{amssymb}
\DeclareMathOperator*{\E}{E}

% \usepackage{amsthm}
% \newtheorem{theorem}{Theorem}

\usepackage{mathtools}

\usepackage{hyperref}
\hypersetup{hidelinks}

% \usepackage{tikz}

\usepackage{geometry}
\geometry{hmargin=3cm,vmargin=3cm}

\begin{document}

\title{The card pairs problem}
\author{Ernest A.\ C.}

\maketitle

The problem was posted on Reddit%
\footnote{\url{https://www.reddit.com/r/statistics/comments/7eginx/}}
in 2017 and is stated as follows:
\begin{quote}
  You have one hundred cards numbered $1,\dotsc,100$. You randomly
  pair all of the cards. Whichever of the pair is a higher number is
  considered to be a ‘winner’. On average, what percentage of cards
  from the upper half ($51,\dotsc,100$) will be considered to be
  ‘winners’?
\end{quote}



Let $A$ and $B$ be the sets of cards in the lower and upper halves of
the deck respectively, and let $W$ be the number of winners in $B$
after randomly pairing all the cards in the deck with one another.  We
are asked to find the expected value of $W / 50$.  Since any card in
$B$ wins when paired with any card in $A$, the largest amount of upper
half winners occurs when every card in $B$ is paired with some card in
$A$, in which case $w = 50$.  Conversely, the least number of upper
half winners occurs when every card in $B$ is paired with another card
in $B$, in which case $w = 25$.  It follows that that $\Pr(W=w) = 0$
for any $w > 50$ and for any $w < 25$.  Therefore we can write the
expected value of $W / 50$ as
\begin{equation}
  \label{eq:expected-value}
  \E(W/50) = \frac{1}{50} \sum_{w=25}^{50} w \Pr(W=w) .
\end{equation}

Our task now is to find $\Pr(W=w)$ for all $w \in \{25,\dotsc,50 \}$.
Let $M$ be the number of cards from the upper half of the deck that
are paired with cards from the lower half, and let $R$ be the number
of cards from the upper half that are paired with other cards from the
upper half.  The number of winners from the upper half of the deck is
\begin{equation*}
  w = M + \frac{R}{2} .
\end{equation*}
Since $M + R = 50$, it is clear that $M$ and $R$ are uniquely
determined by the number of winners from the upper half, $w$.  For
example, if $w = 34$, then $M = 18$ and $R = 32$.

There are ${50 \choose M}$ different $M$-element subsets of $B$.  The
first card in the subset can be paired with any of the $50$ cards in
$A$.  The second card can be paired with any of the $49$ remaining
cards in $A$, and so forth.  The $R$ remaining cards in $B$ are paired
with one another.  The first card can be paired with $R-1$ cards, the
second with $R-3$ cards, and so forth.  The same thing applies to the
$R$ remaining cards in $A$.  Thus, by the multiplication principle,
the number of ways the cards in the deck can be paired with one
another resulting in a particular number of winners from the upper
half of the deck is
\begin{equation*}
  {50 \choose M} \prod^M_{k = 1} (50 - k + 1)
  \left( \prod^{R / 2}_{k = 1} (R - 2k + 1) \right)^2 ,
\end{equation*}
where $R = 100 - 2w$ and $M = 50 - R$.  Dividing by the size of the
sample space
\begin{equation*}
  \prod^{50}_{k = 1} (100 - 2k + 1)
\end{equation*}
we get the probability $\Pr(W = w)$, and substituting in
(\ref{eq:expected-value}) we find that
\begin{equation*}
  \E(W/50) = \frac{37.6262}{50} = 0.7525 .
\end{equation*}
Thus, on average, $75.25\%$ of the cards from the upper half of the
deck will be considered to be winners.

% \begin{figure}[h]
%   \centering
%   % Created by tikzDevice version 0.12 on 2018-07-13 15:53:14
% !TEX encoding = UTF-8 Unicode
\begin{tikzpicture}[x=1pt,y=1pt]
\definecolor{fillColor}{RGB}{255,255,255}
\path[use as bounding box,fill=fillColor,fill opacity=0.00] (0,0) rectangle (289.08,216.81);
\begin{scope}
\path[clip] ( 49.20, 61.20) rectangle (263.88,167.61);
\definecolor{drawColor}{RGB}{0,0,0}

\path[draw=drawColor,line width= 0.4pt,dash pattern=on 1pt off 3pt ,line join=round,line cap=round] ( 57.15, 65.14) -- ( 57.15, 65.14);

\path[draw=drawColor,line width= 0.4pt,dash pattern=on 1pt off 3pt ,line join=round,line cap=round] ( 65.10, 65.14) -- ( 65.10, 65.14);

\path[draw=drawColor,line width= 0.4pt,dash pattern=on 1pt off 3pt ,line join=round,line cap=round] ( 73.05, 65.14) -- ( 73.05, 65.14);

\path[draw=drawColor,line width= 0.4pt,dash pattern=on 1pt off 3pt ,line join=round,line cap=round] ( 81.00, 65.14) -- ( 81.00, 65.14);

\path[draw=drawColor,line width= 0.4pt,dash pattern=on 1pt off 3pt ,line join=round,line cap=round] ( 88.96, 65.14) -- ( 88.96, 65.14);

\path[draw=drawColor,line width= 0.4pt,dash pattern=on 1pt off 3pt ,line join=round,line cap=round] ( 96.91, 65.14) -- ( 96.91, 65.15);

\path[draw=drawColor,line width= 0.4pt,dash pattern=on 1pt off 3pt ,line join=round,line cap=round] (104.86, 65.14) -- (104.86, 65.22);

\path[draw=drawColor,line width= 0.4pt,dash pattern=on 1pt off 3pt ,line join=round,line cap=round] (112.81, 65.14) -- (112.81, 65.76);

\path[draw=drawColor,line width= 0.4pt,dash pattern=on 1pt off 3pt ,line join=round,line cap=round] (120.76, 65.14) -- (120.76, 68.50);

\path[draw=drawColor,line width= 0.4pt,dash pattern=on 1pt off 3pt ,line join=round,line cap=round] (128.71, 65.14) -- (128.71, 77.85);

\path[draw=drawColor,line width= 0.4pt,dash pattern=on 1pt off 3pt ,line join=round,line cap=round] (136.66, 65.14) -- (136.66, 99.38);

\path[draw=drawColor,line width= 0.4pt,dash pattern=on 1pt off 3pt ,line join=round,line cap=round] (144.61, 65.14) -- (144.61,131.84);

\path[draw=drawColor,line width= 0.4pt,dash pattern=on 1pt off 3pt ,line join=round,line cap=round] (152.56, 65.14) -- (152.56,159.88);

\path[draw=drawColor,line width= 0.4pt,dash pattern=on 1pt off 3pt ,line join=round,line cap=round] (160.52, 65.14) -- (160.52,163.67);

\path[draw=drawColor,line width= 0.4pt,dash pattern=on 1pt off 3pt ,line join=round,line cap=round] (168.47, 65.14) -- (168.47,140.21);

\path[draw=drawColor,line width= 0.4pt,dash pattern=on 1pt off 3pt ,line join=round,line cap=round] (176.42, 65.14) -- (176.42,106.90);

\path[draw=drawColor,line width= 0.4pt,dash pattern=on 1pt off 3pt ,line join=round,line cap=round] (184.37, 65.14) -- (184.37, 81.98);

\path[draw=drawColor,line width= 0.4pt,dash pattern=on 1pt off 3pt ,line join=round,line cap=round] (192.32, 65.14) -- (192.32, 70.00);

\path[draw=drawColor,line width= 0.4pt,dash pattern=on 1pt off 3pt ,line join=round,line cap=round] (200.27, 65.14) -- (200.27, 66.13);

\path[draw=drawColor,line width= 0.4pt,dash pattern=on 1pt off 3pt ,line join=round,line cap=round] (208.22, 65.14) -- (208.22, 65.28);

\path[draw=drawColor,line width= 0.4pt,dash pattern=on 1pt off 3pt ,line join=round,line cap=round] (216.17, 65.14) -- (216.17, 65.15);

\path[draw=drawColor,line width= 0.4pt,dash pattern=on 1pt off 3pt ,line join=round,line cap=round] (224.12, 65.14) -- (224.12, 65.14);

\path[draw=drawColor,line width= 0.4pt,dash pattern=on 1pt off 3pt ,line join=round,line cap=round] (232.08, 65.14) -- (232.08, 65.14);

\path[draw=drawColor,line width= 0.4pt,dash pattern=on 1pt off 3pt ,line join=round,line cap=round] (240.03, 65.14) -- (240.03, 65.14);

\path[draw=drawColor,line width= 0.4pt,dash pattern=on 1pt off 3pt ,line join=round,line cap=round] (247.98, 65.14) -- (247.98, 65.14);

\path[draw=drawColor,line width= 0.4pt,dash pattern=on 1pt off 3pt ,line join=round,line cap=round] (255.93, 65.14) -- (255.93, 65.14);
\end{scope}
\begin{scope}
\path[clip] (  0.00,  0.00) rectangle (289.08,216.81);
\definecolor{drawColor}{RGB}{0,0,0}

\path[draw=drawColor,line width= 0.4pt,line join=round,line cap=round] ( 57.15, 61.20) -- (255.93, 61.20);

\path[draw=drawColor,line width= 0.4pt,line join=round,line cap=round] ( 57.15, 61.20) -- ( 57.15, 55.20);

\path[draw=drawColor,line width= 0.4pt,line join=round,line cap=round] ( 96.91, 61.20) -- ( 96.91, 55.20);

\path[draw=drawColor,line width= 0.4pt,line join=round,line cap=round] (136.66, 61.20) -- (136.66, 55.20);

\path[draw=drawColor,line width= 0.4pt,line join=round,line cap=round] (176.42, 61.20) -- (176.42, 55.20);

\path[draw=drawColor,line width= 0.4pt,line join=round,line cap=round] (216.17, 61.20) -- (216.17, 55.20);

\path[draw=drawColor,line width= 0.4pt,line join=round,line cap=round] (255.93, 61.20) -- (255.93, 55.20);

\node[text=drawColor,anchor=base,inner sep=0pt, outer sep=0pt, scale=  1.00] at ( 57.15, 39.60) {25};

\node[text=drawColor,anchor=base,inner sep=0pt, outer sep=0pt, scale=  1.00] at ( 96.91, 39.60) {30};

\node[text=drawColor,anchor=base,inner sep=0pt, outer sep=0pt, scale=  1.00] at (136.66, 39.60) {35};

\node[text=drawColor,anchor=base,inner sep=0pt, outer sep=0pt, scale=  1.00] at (176.42, 39.60) {40};

\node[text=drawColor,anchor=base,inner sep=0pt, outer sep=0pt, scale=  1.00] at (216.17, 39.60) {45};

\node[text=drawColor,anchor=base,inner sep=0pt, outer sep=0pt, scale=  1.00] at (255.93, 39.60) {50};

\path[draw=drawColor,line width= 0.4pt,line join=round,line cap=round] ( 49.20, 65.14) -- ( 49.20,155.27);

\path[draw=drawColor,line width= 0.4pt,line join=round,line cap=round] ( 49.20, 65.14) -- ( 43.20, 65.14);

\path[draw=drawColor,line width= 0.4pt,line join=round,line cap=round] ( 49.20, 87.67) -- ( 43.20, 87.67);

\path[draw=drawColor,line width= 0.4pt,line join=round,line cap=round] ( 49.20,110.21) -- ( 43.20,110.21);

\path[draw=drawColor,line width= 0.4pt,line join=round,line cap=round] ( 49.20,132.74) -- ( 43.20,132.74);

\path[draw=drawColor,line width= 0.4pt,line join=round,line cap=round] ( 49.20,155.27) -- ( 43.20,155.27);

\node[text=drawColor,rotate= 90.00,anchor=base,inner sep=0pt, outer sep=0pt, scale=  1.00] at ( 34.80, 65.14) {0.00};

\node[text=drawColor,rotate= 90.00,anchor=base,inner sep=0pt, outer sep=0pt, scale=  1.00] at ( 34.80,110.21) {0.10};

\node[text=drawColor,rotate= 90.00,anchor=base,inner sep=0pt, outer sep=0pt, scale=  1.00] at ( 34.80,155.27) {0.20};
\end{scope}
\begin{scope}
\path[clip] (  0.00,  0.00) rectangle (289.08,216.81);
\definecolor{drawColor}{RGB}{0,0,0}

\node[text=drawColor,anchor=base,inner sep=0pt, outer sep=0pt, scale=  1.00] at (156.54, 15.60) {W};

\node[text=drawColor,rotate= 90.00,anchor=base,inner sep=0pt, outer sep=0pt, scale=  1.00] at ( 10.80,114.41) {Probability};
\end{scope}
\begin{scope}
\path[clip] ( 49.20, 61.20) rectangle (263.88,167.61);
\definecolor{drawColor}{RGB}{0,0,0}
\definecolor{fillColor}{RGB}{0,0,0}

\path[draw=drawColor,line width= 0.4pt,line join=round,line cap=round,fill=fillColor] ( 57.15, 65.14) circle (  1.50);

\path[draw=drawColor,line width= 0.4pt,line join=round,line cap=round,fill=fillColor] ( 65.10, 65.14) circle (  1.50);

\path[draw=drawColor,line width= 0.4pt,line join=round,line cap=round,fill=fillColor] ( 73.05, 65.14) circle (  1.50);

\path[draw=drawColor,line width= 0.4pt,line join=round,line cap=round,fill=fillColor] ( 81.00, 65.14) circle (  1.50);

\path[draw=drawColor,line width= 0.4pt,line join=round,line cap=round,fill=fillColor] ( 88.96, 65.14) circle (  1.50);

\path[draw=drawColor,line width= 0.4pt,line join=round,line cap=round,fill=fillColor] ( 96.91, 65.15) circle (  1.50);

\path[draw=drawColor,line width= 0.4pt,line join=round,line cap=round,fill=fillColor] (104.86, 65.22) circle (  1.50);

\path[draw=drawColor,line width= 0.4pt,line join=round,line cap=round,fill=fillColor] (112.81, 65.76) circle (  1.50);

\path[draw=drawColor,line width= 0.4pt,line join=round,line cap=round,fill=fillColor] (120.76, 68.50) circle (  1.50);

\path[draw=drawColor,line width= 0.4pt,line join=round,line cap=round,fill=fillColor] (128.71, 77.85) circle (  1.50);

\path[draw=drawColor,line width= 0.4pt,line join=round,line cap=round,fill=fillColor] (136.66, 99.38) circle (  1.50);

\path[draw=drawColor,line width= 0.4pt,line join=round,line cap=round,fill=fillColor] (144.61,131.84) circle (  1.50);

\path[draw=drawColor,line width= 0.4pt,line join=round,line cap=round,fill=fillColor] (152.56,159.88) circle (  1.50);

\path[draw=drawColor,line width= 0.4pt,line join=round,line cap=round,fill=fillColor] (160.52,163.67) circle (  1.50);

\path[draw=drawColor,line width= 0.4pt,line join=round,line cap=round,fill=fillColor] (168.47,140.21) circle (  1.50);

\path[draw=drawColor,line width= 0.4pt,line join=round,line cap=round,fill=fillColor] (176.42,106.90) circle (  1.50);

\path[draw=drawColor,line width= 0.4pt,line join=round,line cap=round,fill=fillColor] (184.37, 81.98) circle (  1.50);

\path[draw=drawColor,line width= 0.4pt,line join=round,line cap=round,fill=fillColor] (192.32, 70.00) circle (  1.50);

\path[draw=drawColor,line width= 0.4pt,line join=round,line cap=round,fill=fillColor] (200.27, 66.13) circle (  1.50);

\path[draw=drawColor,line width= 0.4pt,line join=round,line cap=round,fill=fillColor] (208.22, 65.28) circle (  1.50);

\path[draw=drawColor,line width= 0.4pt,line join=round,line cap=round,fill=fillColor] (216.17, 65.15) circle (  1.50);

\path[draw=drawColor,line width= 0.4pt,line join=round,line cap=round,fill=fillColor] (224.12, 65.14) circle (  1.50);

\path[draw=drawColor,line width= 0.4pt,line join=round,line cap=round,fill=fillColor] (232.08, 65.14) circle (  1.50);

\path[draw=drawColor,line width= 0.4pt,line join=round,line cap=round,fill=fillColor] (240.03, 65.14) circle (  1.50);

\path[draw=drawColor,line width= 0.4pt,line join=round,line cap=round,fill=fillColor] (247.98, 65.14) circle (  1.50);

\path[draw=drawColor,line width= 0.4pt,line join=round,line cap=round,fill=fillColor] (255.93, 65.14) circle (  1.50);
\end{scope}
\end{tikzpicture}

%   \caption{Probability distribution of $W$.}
%   \label{fig:pmf}
% \end{figure}

\end{document}